\addcontentsline{toc}{chapter}{Resumen}
\chapter*{Resumen}

  Muchos problemas de la vida real requieren de la optimizaci\'on de dos o m\'as objetivos a la vez, los cuales, 
  est\'an en conflicto unos con otros al mismo tiempo. Por lo general, los problemas de optimizaci\'on multi-objetivo
  no tienen una soluci\'on \'unica sino un conjunto de soluciones que representan los diferentes compromisos entre 
  los objetivos. 

  La mayor\'ia de los algoritmos evolutivos multi-objetivo utilizan la dominancia de Pareto como criterio de
  selecci\'on. Si bien este mecanismo es efectivo en problemas con dos o tres objetivos, no es escalable ya 
  que la proporci\'on de soluciones incomparables que se generan crece r\'apidamente al aumentar el n\'umero de objetivos.

  En este trabajo de tesis se utiliza la optimizaci\'on mediante c\'umulos de part\'iculas 
  (PSO), la cual es una t\'ecnica evolutiva inspirada en el comportamiento social de los bancos de peces. Se plantea 
  una forma de hacer que un algoritmo de c\'umulo de part\'iculas resuelva problemas de optimizaci\'on multi-objetivo.
  Al optimizador propuesto se le integra como mecanismo de selecci\'on la m\'etrica denominada hipervolumen, 
  a fin de obtener a los mejores l\'ideres o gu\'ias en su proceso de b\'usqueda. Nuestra propuesta obtiene buenos resultados 
  para problemas con dos y tres objetivos y presenta resultados prometedores para problemas con muchos objetivos (cuatro o 
  m\'as).


%\addcontentsline{toc}{chapter}{Abstract}
\chapter*{Abstract}

  Many real-life problems require optimization the simultaneous of two or more conflicting objectives. 
  Usually, multiobjective optimization problems do not have a single solution but a set of them,
  representing the trade-offs among the objetives.

  Most multiobjective evolutionary algorithms \DIFdelbegin \DIFdel{using }\DIFdelend \DIFaddbegin \DIFadd{use }\DIFaddend Pareto dominance as their selection criterion. This 
  \DIFdelbegin \DIFdel{mechanisms }\DIFdelend \DIFaddbegin \DIFadd{mechanism }\DIFaddend is effective in problems having two or three objectives but it does not \DIFdelbegin \DIFdel{property }\DIFdelend \DIFaddbegin \DIFadd{properly }\DIFaddend scale since 
  number of incomparable solutions that this selection criterion generates rapidly grows as \DIFdelbegin \DIFdel{we increase }\DIFdelend the
  number of objectives \DIFaddbegin \DIFadd{increases}\DIFaddend .

  In this thesis we use Particle Swarm Optimization (PSO), which is an evolutionary technique inspired \DIFdelbegin \DIFdel{on }\DIFdelend \DIFaddbegin \DIFadd{in }\DIFaddend the social 
  behavior of fish schools. We present here a way to adapt \DIFdelbegin \DIFdel{PSO s }\DIFdelend \DIFaddbegin \DIFadd{PSOs }\DIFaddend that it can solve multiobjetive optimization problems.
  The proposed approach adopts as its selection mechanism, a performance measure called hypervolume to find the best leaders
  during the search process. Our proposed approach was able to obtain good results for problems with two and three 
  objectives and presented promising results for problems with many objectives (more than three).