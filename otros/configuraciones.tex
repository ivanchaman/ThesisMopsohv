%%%%%%%%%Comandos para titulo
\newcommand{\bigrule}{\titlerule[0.5mm]}
\titleformat{\chapter}[display] % cambiamos el formato de los capítulos
{\bfseries\Huge} % por defecto se usarán caracteres de tamaño \Huge en negrita
{% contenido de la etiqueta
\titlerule % línea horizontal
\filleft % texto alineado a la derecha
\Large\chaptertitlename\ % "Capítulo" o "Apéndice" en tamaño \Large en lugar de \Huge
\Large\thechapter} % número de capítulo en tamaño \Large
{0mm} % espacio mínimo entre etiqueta y cuerpo
{\filleft} % texto del cuerpo alineado a la derecha
[\vspace{0.5mm} \bigrule] % después del cuerpo, dejar espacio vertical y trazar línea horizontal gruesa
%%%%%%%%%%
%% Es para poner los márgenes en una impresión
%% a ambas caras de la hoja
\renewcommand{\tablename}{Tabla}
\renewcommand{\sin}{\operatorname{sen}}
\decimalpoint

%confguraciones para codigos en C
\definecolor{gray97}{gray}{.97}
\definecolor{gray75}{gray}{.75}
\definecolor{gray45}{gray}{.45}
 
\lstset{ 
    frame=Ltb,
     framerule=0pt,
     aboveskip=0.5cm,
     framextopmargin=3pt,
     framexbottommargin=3pt,
     framexleftmargin=0.4cm,
     framesep=0pt,
     rulesep=.4pt,
     backgroundcolor=\color{gray97},
     rulesepcolor=\color{black},
     %
     stringstyle=\ttfamily,
     showstringspaces = fase,
    % basicstyle=\small\ttfamily,
    basicstyle=\footnotesize,
     commentstyle=\color{gray45},
     keywordstyle=\bfseries,
     %
     numbers=left,
     numbersep=15pt,
     numberstyle=\tiny,
     numberfirstline = false,
     breaklines=true,
   }
 
% minimizar fragmentado de listados
\lstnewenvironment{listing}[1][]
   {\lstset{#1}\pagebreak[0]}{\pagebreak[0]}
 
\lstdefinestyle{consola}
   {basicstyle=\scriptsize\bf\ttfamily,
    backgroundcolor=\color{gray75},
   }
 
\lstdefinestyle{C}
   {language=C,} 