\begin{chapter}{Conclusiones y trabajo futuro}
  
  \section{Resumen}
  
  El objetivo principal de este trabajo de tesis fue proponer un nuevo algoritmo multi-objetivo basado los c\'umulos de 
  part\'iculas la cual mostrara ser competitiva con respecto a algoritmos evolutivos multiobjetivo (AEMOs) representativos del estado 
  del arte.

  Para llevar a cabo este objetivo se realiz\'o un estudio de las metaheur\'isticas multi-objetivo basadas en 
  c\'umulos de part\'iculas que existen actualmente. Posteriormente, se realiz\'o un estudio del denominado hipervolumen 
  para poder incorporarlo en el mecanismo de selecci\'on de nuestra metaheur\'istica. A este respecto, se identificaron
  varias formas de utilizar el hipervolumen como mecanismo de selecci\'on, sobresaliendo
  el uso del algoritmo HypE que usa un m\'etodo para aproximar las contribuciones al hipervolumen. 
  Como producto de este an\'alisis se propuso un algoritmo de c\'umulos de part\'iculas multi-objetivo que aprovecha las contribuciones al 
  hipervolumen para seleccionar la mejor gu\'ia global a fin de generar soluciones potenciales no dominadas.

  El algoritmo propuesto fue comparado con respecto a NSGA-II, SMS-EMOA y MOPSOcd. La evaluaci\'on se realiz\'o utilizando un conjunto 
  de problemas multi-objetivo que re\'unen diferentes 
  caracter\'isticas que causan dificultades a un algoritmo evolutivo multi-objetivo.
  
  \section{Conclusiones}
  
  A partir de los experimentos realizados se desprenden las siguientes conclusiones:

  \begin{itemize}
  \item Una de las principales dificultades en extender el algoritmo del PSO a una versi\'on multi-objetivo, es encontrar la mejor 
  forma de seleccionar al l\'ider de cada part\'icula en el c\'umulo. Esta dificultad radica en que no hay una noci\'on clara sobre 
  c\'omo definir qui\'en es el mejor alcanzado hasta el momento (\textbf{\textit{pBest}}) y el mejor 
  alcanzado por toda la poblaci\'on (\textbf{\textit{gBest}}). En los problemas de optimizaci\'on multi-objetivo, 
  todas las soluciones no dominadas son igualmente buenas, por lo que cualquiera de ellas puede adoptarse como l\'ider. Por tanto, el 
  uso del hipervolumen nos permiti\'o seleccionar al mejor l\'ider alcanzado de un conjunto de soluciones no dominadas. La selecci\'on
  se realiza conforme a su contribuci\'on al hipervolumen, es decir, aquella part\'icula que contribuye m\'as al valor del hipervolumen
  es seleccionada como l\'ider.
  
  \item El uso del hipervolumen para seleccionar al l\'ider en nuestro algoritmo muestra una mejora con respecto al algoritmo MOPSOcd. 
  
  \item La forma de seleccionar al conjunto de part\'iculas que influyen en el entorno social  del algoritmo ($pBest$) mejora 
  considerablemente la b\'usqueda de nuevas soluciones no dominadas. Es mejor utilizar el archivo o la poblaci\'on
  secundaria, en lugar de la poblacion primaria, para actualizar el $pBest$ del algoritmo propuesto.
  
  \item La contribuci\'on al hipervolumen es un valor que se basa en la m\'etrica del hipervolumen. Por lo que, resulta ser muy costoso 
  computacionalmente hablando. Sin embargo, el uso de algoritmos que hacen uso de metodos 
  de muestreo para calcular la contribuci\'on al hipervolumen resultan ser m\'as eficientes. El uso de HypE 
  es un algoritmo que utiliza el m\'etodo de Monte Carlo y muestra ser m\'as eficiente al calcular las contribuciones al hipervolumen.
  
  \item El uso del algoritmo de HypE para calcular las contribuciones al hipervolumen permite aumentar el n\'umero de generaciones 
  del algoritmo y el n\'umero de objetivos del problema.
  
  \item Debido a que se utiliza una aproximaci\'on de la contribuci\'on al hipervolumen, se decidi\'o utilizar un operador de 
  turbulencia para evitar quedar atrapado en frentes de Pareto locales.
  
  \item Los resultados obtenidos para los primeros tres problemas ZDT son similares. Para el problema ZDT4 es mejor que los resultados 
  obtenidos por el SMS-EMOA y marginalmente peores que los del NSGA-II. As\'i mismo, nuestros resultados son mejores que los de 
  MOPSOcd. Para el problema ZDT6 los resultados son ligeramente mejores que los de SMS-EMOA y ligeramente peores que los de NSGA-II, 
  siendo mucho mejores que los del MOPSOcd.
  
  \item Los resultados obtenidos por nuestra propuesta para los problemas DTLZ1, DTLZ3 y DTLZ6 son mejores que los del MOPSOcd.
  Sin embargo, nuestro algoritmo no pudo obtener buenas aproximaciones al frente de Pareto real, y nuestros resultados fueron dominados
  por los de SMS-EMOA y NSGA-II. Los resultados para los dem\'as problemas (DTLZ2, DTLZ4, DTLZ5 y DTLZ7) muestran que en la mayoria 
  de los casos, nuestra propuesta obtiene soluciones que dominan a las de los algoritmos NSGA-II, SMS-EMOA y MOPSOcd.
  
  \item Los resultados muestran que en la mayoria de los problemas (2 y 3 objetivos) de nuestra propuesta es competitiva
  con respecto a los algoritmos NSGA-II, SMS-EMOA y MOPSOcd.
  
  \item Nuestro algoritmo mantiene un buen desempe\~no al aumentar el n\'umero de objetivos del problema, manteniendo un costo computacional 
  razonable. Esto hace que nuestro algoritmo sea competitivo con respecto a los algoritmos NSGA-II, al SMS-EMOA y al MOPSOcd.
  \end{itemize}
  
  \section{Trabajo futuro}
  
  Existen diversas formas de poder dise\~nar un algoritmo multi-objetivo con base en la metaheur\'istica de c\'umulos de 
  part\'iculas, pues es posible usar diferentes tipos de topolog\'ias o conexiones para que las part\'iculas interact\'uen o 
  influyan entre ellas a fin de para generar buenas aproximaciones hacia frente de Pareto.
 
  En particular se puedr\'ian realizar las siguientes extensiones a nuestro algoritmo:
  
  \begin{itemize}
  \item Utilizar modelos de configuraci\'on diferentes al modelo completo que se us\'o aqu\'i 
   (modelo cognitivo, modelo social o modelo social exclusivo). 
  \item Utilizar otros apectos avanzados que pueden acelerar la convergencia del algoritmo (factor de constricci\'on, factor de inercia 
  adaptable o control de velocidad).
  
 \item Seleccionar a los gu\'ias locales de otra manera, de tal forma, que afectan la parte cognitiva del algoritmo permitiendo
 generar nuevas soluciones.
 
 \item Seleccionar un mejor estimador de densidad para hacer el reemplazo de los soluciones no dominadas en la poblaci\'on secundaria.
 
 \end{itemize}  
\end{chapter}