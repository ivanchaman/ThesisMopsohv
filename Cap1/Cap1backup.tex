\addcontentsline{toc}{chapter}{Introducci\'on}
\chapter*{Introducci\'on}
%\minitoc
  Se proporciona una breve introducci\'on a este trabajo de tesis, as\'i como el planteamiento del problema 
  a resolverse y los objetivos planteados. En la parte final se propociona una breve descripci\'on 
  del contenido de cada cap\'itulo de este trabajo de tesis.

\section*{Antecedentes}

  Los algoritmos evolutivos pueden verse como una analog\'ia del mecanismo de selecci\'on natural, cuyo principal objetivo es 
  simular el proceso evolutivo en una computadora y usarlo para resolver problemas de optimizaci\'on. El uso de 
  algoritmos evolutivos se ha extendido a un gran n\'umero de dominios debido, sobre todo, a su simplicidad conceptual
  y facilidad de uso.
  
  Un problema de optimizaci\'on multi-objetivo difiere de un problema de  optimizaci\'on mono-objetivo en la cantidad de 
  objetivos a optimizarse. Mientras que en un problema de optimizaci\'on mono-objetivo se busca la mejor soluci\'on posible 
  en todo el espacio de b\'usqueda, en un problema multi-objetivo, se tienen varios objetivos (posiblemente 
  en conflicto), y normalmente no hay una soluci\'on \'unica sino un conjunto de soluciones compromiso. Por lo tanto, en este 
  \'ultimo caso se requiere elegir una soluci\'on entre varias, a partir de las preferencias del usuario.
  
  La principal motivaci\'on para usar algoritmos evolutivos para resolver problemas de optimizaci\'on multi-objetivo  
  es explotar su poblaci\'on, de manera que podamos generar varios miembros del conjunto de \'optimos de Pareto en una sola 
  ejecuci\'on del algoritmo, en lugar de tener que realizar una serie de ejecuciones por separado como en el caso de las 
  t\'ecnicas de programaci\'on matem\'atica. Adem\'as, los algoritmos evolutivos son menos susceptibles a la forma o la 
  continuidad del frente de Pareto, mientras que las t\'ecnias de programaci\'on matem\'atica suelen tener dificultades 
  para lidiar con frentes de Pareto desconectados y no convexos.

\section*{Planteamiento del problema}

El uso de esquemas de selecci\'on basados en la optimalidad de Pareto presenta diversas limitantes, dentro de las que destaca
su pobre escalabilidad cuando se aumenta la cantidad de funciones objetivo. En a\~nos recientes, se han planteado diferentes 
mecanismos de selecci\'on para algoritmos evolutivos multi-objetivo que no se basan en la optimalidad de Pareto. Una de las 
alternativas m\'as prometedoras es el uso de indicadores de desempe\~no para seleccionar soluciones. De entre los posibles 
indicadores que pueden utilizarse, el hipervolumen (o m\'etrica $\mathcal{S}$) es, sin duda el m\'as popular debido, sobre 
todo, a sus propiedades matem\'aticas (se ha podido demostrar que maximizar el hipervolumen conduce a lograr convergencia). 
En este trabajo se plantea incorporar en un algoritmo de c\'umulos de part\'iculas un mecanismo  basado en el indicador de 
hipervolumen para seleccionar a los mejores l\'ideres que conducir\'an la b\'usqueda. El nuevo algoritmo deber\'a ser competitivo
con respecto a algoritmos evolutivos multi-objetivo del estado del arte.

\section*{Objetivos}

\subsubsection*{Objetivo general}

Desarrollar un algoritmo multi-objetivo con base en la metaheur\'istica conocida como c\'umulos de part\'iculas, el cual utilice el 
hipervolumen en su mecanismo de selecci\'on y sea competitivo con otros algoritmos evolutivos multi-objetivo.

\subsubsection*{Objetivos espec\'ificos}

\begin{itemize}
 \item Estudiar las diferentes maneras de implementar el c\'alculo del hipervolumen (de manera exacta o aproximada).
 \item Analizar diferentes algoritmos basados en c\'umulos de part\'iculas para establecer las diferencias y 
      similitudes entre ellos.
 \item Dise\~nar un mecanismo de selecci\'on que aproveche las caracter\'isticas del hipervolumen, e incorpor\'arselo 
 a un algoritmo multi-objetivo basado en c\'umulos de part\'iculas.
 \item Comparar el desempe\~no de las soluciones del algoritmo desarrallado con respecto a las soluciones de otros algoritmos
  utilizando un conjunto diverso de problemas de la literatura especializada.
\end{itemize}

\section*{Organizaci\'on}

La organizaci\'on de este trabajo de tesis es el siguiente:

\begin{itemize}
 \item  En el cap\'itulo 1 se describen algunos conceptos b\'asicos sobre algoritmos evolutivos, optimizaci\'on multiobjetivo, 
 la metaheur\'istica de c\'umulos de part\'iculas y las carater\'isticas del indicador de hipervolumen que servir\'an como 
 marco te\'orico  para el resto del trabajo de tesis. Adem\'as, se proporciona una revision del estado del arte sobre los 
 algoritmos evolutivos multi-objetivo.
 \item En el cap\'itulo 2 se describe el algoritmo de c\'umulos de part\'iculas multi-objetivo propuesto en este trabajo de 
 tesis, cuyo mecanismo de selecci\'on est\'a basado en la contribuci\'on al hipervolumen y adopta un operador de turbulencia.
 \item En el cap\'itulo 3 se compara el algoritmo propuesto con respecto a otros algoritmos evolutivos multi-objetivo tomados
 de la literatura especializada  utilizando los indicadores de calidad descritos en el cap\'itulo 1.
 \item En el cap\'itulo 4 se proporcionan las conclusiones y el trabajo futuro.
\end{itemize}
